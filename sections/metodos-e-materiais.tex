\section{Métodos e Materiais}\label{sec:metodos-e-materiais}


Será realizado um estudo de caso, baseado em pesquisa experimental pois há um alto nível de controle da situação e podem-se isolar todas as estruturas de qualquer interferência do meio exterior, gerando maior confiabilidade nos resultados.

Será realizada uma análise quantitativa da quantidade de coincidências entre trilhas e detecções de forma a avaliar e validar a eficiência em relação à detecção de falsos positivos indicados pela solução. Esses dados serão obtidos fazendo um comparativo entre o cenário real e os dados indicados pela solução. Será utilizada para isso a técnica de estudo experimental \textit{In Virtuo}

A avaliação se dará no que segue:

\begin{enumerate}
\item Adicionar o Kinect e os sensores em pontos estratégicos da sala;
\item Iniciar o sistema produto da solução;
\item Avaliar entrada e saída de um grupo distinto de pessoas em momentos distintos e ao mesmo tempo;
\item Comparar os dados obtidos pela solução com o ocorrido nas cenas durante a avaliação
\item Calcular a eficiência de acordo com os dados obtidos dessa comparação.
\end{enumerate}


%\subsection{Passo a Passo Latex}\label{sec:passo-a-passo}
Esta subseção apresenta o passo a passo que deve ser seguido para utilizar o sharelatex com o template da ECDU.

\begin{enumerate}
\item Criar projeto no ShareLatex com template IEEE Conference
\item Coloque as suas bibliografias em um arquivo separado. Para isso crie o arquivo .bib. Veja exemplo "\\bibliography{references.bib}
\\bibliographystyle{IEEEtran}" no texto. Remova a parte de referencias que já vem no template
\item No lugar dos dois itens anteriores, você pode fazer o download do projeto ECDU-TCC-Modelo e depois fazer o upload para sua conta, ou fazer a cópia direta (Opção Copy Project no menu esquerdo superior).
\item Mudar para português. Veja o código necessário aqui https://pt.sharelatex.com/learn/Portuguese. Esse código já está disponível no arquivo main.tex. Ele está entre os comentários "Início do código para entender português" e "Fim do código para entender português"
\item Corretor Ortográfico para português. Mude as configurações do seu projeto para Português do Brasil: Menu -> Spell Check -> Portuguese (Brazilian)
\end{enumerate}

