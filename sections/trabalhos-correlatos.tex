\section{Trabalhos Relacionados}\label{sec:trabalhos-relacionados}

A pesquisa sobre o monitoramento de vídeo dinâmico tornou-se mais madura nos últimos anos. Na literatura especializada, o conjunto de trabalhos que utilizam dados do
\textit{Microsoft Kinect} para o reconhecimento BD de faces é bastante reduzido. Os trabalhos mais interessantes são os de Goswami et al. \cite{goswami2013rgb} e Li et al. \cite{li2013using}.

Goswami et al. \cite{goswami2013rgb} geram mapas de entropia e saliência visual para os dados gerados pelo \textit{Kinect}. Para as imagens RGB ambos os mapas são gerados, para os dados de profundidade somente o mapa de entropia é gerado. O descritor da face é gerado concatenando os HOGs (Histograma de Gradientes Orientados) de diferentes fragmentos de ambos os mapas. Para a classificação dos sujeitos e utilizado o Classificador RDF (Floresta Randômica de Decisão).
 
Li et al. \cite{goswami2013rgb} também utilizam tanto os dados de profundidade quanto as imagens RGB para fazer detecção 3D de faces. Como primeiro passo do método os dados em RGB passam pela transformada DCS (Espaço de Cor Discriminante) para aumentar seu poder discriminativo. Como possuem três canais eles são empilhados. As imagens de profundidade são baseadas nas nuvens de pontos gerada pelo  \textit{Kinect}. Para melhorar a densidade elas são submetidas a um processo de Preenchimento Simétrico ( \textit{Symmetric Filling} no original).
 
Preenchimento Simétrico e um processo onde a nuvem de pontos de uma face é espelhada e, cada ponto na face original, é comparado com um ponto na nuvem espelhada. Caso a distância entre os dois pontos seja menor que um  \textit{threshold t} então o ponto da nuvem espelhada passa a fazer parte da nuvem original. Devido às propriedades simétricas da face esse processo tende a melhorar a qualidade dos dados gerados pelo \textit{Kinect}.
 
Após gerar as imagens de profundidade baseadas nas novas nuvens de pontos o classificador SRC (Classificador de Representação Esparsa) é utilizado para classificação dos sujeitos.
 
Mandeljc et al. \cite{mandeljc2012tracking} propuseram um sistema usando Banda ultralarga (\textit{Ultra wideband - UWB}) e vídeo para o posicionamento de pessoas em ambientes \textit{indoor}. Embora este sistema tenha um bom efeito de posicionamento, equipamentos UWB e múltiplas câmeras são obrigados a ser instalado no local, tornando o custo de construção do sistema é muito alto.

Wang et al. \cite{wang2011flexible,wang2011rfid} criaram um sistema de monitoramento em tempo real e posicionamento de pessoas em ambientes \textit{indoor} combinando imagem com  tecnologias RFID. No entanto, encontra-se experimental e continuam a verificar sobreposição de pessoas e problemas de blindagem, bem como os resultados de posicionamento são susceptíveis de serem influenciadas por altura e tipo do corpo, assim, o efeito implementação requer melhorias.

A utilização do \textit{Kinect} resolve os problemas anteriormente referidos. Com o \textit{Kinect} pode-se extrair com precisão as pessoas e a identificação não terá falhas resultantes de diferentes alturas, tipos de corpo e cores de pele. Seu o chip de processamento de imagem incorporado pode aumentar significativamente a eficiência do processamento de imagem, o que a torna muito aplicável à implementação de um sistema de posicionamento de pessoas \textit{indoor}.

Alguns estudos utilizam o \textit{Kinect} para projetar sistemas de posicionamento de pessoas. Schindhelm \cite{schindhelm2012evaluating} mostrou que o \textit{Kinect} foi muito aplicável no posicionamento \textit{indoor} de edifícios públicos. Nakano et al. \cite{nakano2012kinect} também propôs usar o \textit{Kinect} para o posicionamento de pessoas \textit{indoor}. No entanto, os sistemas acima referidos não têm a função de identificação de pessoas. 

Sung et al. \cite{sung2012unstructured} propõe um sistema de detecção de atividade e reconhecimento usando um sensor de RGBD \textit{Kinect}. Primeiramente é capturada a natureza hierárquica usando um modelo gráfico probabilístico hierárquico, especificamente modelo de entropia máxima de Markov (\textit{maximum entropy Markov model } - Memm) de duas camadas. Mesmo com esse modelo estruturado no lugar, pessoas diferentes realizam diferentes tarefas em velocidades diferentes, e qualquer modelo gráfico único, provavelmente, não consegue captar essa variação. Para superar esse problema, é apresentado um método de seleção de estrutura gráfica \textit{on-the-fly} que pode se adaptar a variações na velocidade automaticamente. O ultimo passo, para capturar as características significativas de uma pessoa, é utilizando o sistema de rastreamento de esqueleto \textit{PrimeSense} \cite{primeSence2014} em combinação com histograma de gradientes orientados. Esse trabalho tem seu foco, porém, na detecção de pessoas para extração de informações de atividades rotineiras de um certo indivíduo.


Há um grande corpo de outros trabalhos sobre o reconhecimento de atividade humana. Uma abordagem comum é usar características espaço-temporais para modelar pontos de interesse em vídeo \cite{dollar2005behavior,laptev2003space}.



Lu Xia et al. \cite{xia2011human} apresentam um método baseado em modelo para detecção humana a partir de imagens de profundidade. este método detecta as pessoas usando informações de profundidade obtidas pelo \textit{Kinect} em ambientes internos. utilizando um processo de detecção de cabeça de 2 estágios, que inclui um detector de borda 2D e um detector de forma 3D para então utilizar as informações de borda e as informações de mudança de profundidade relacional na imagem de profundidade. Também propõe um método de segmentação para segmentar a figura dos objetos de fundo que lhe são anexados e extrair o contorno geral do assunto com precisão. O método é avaliado em um conjunto de dados 3D previamente obtido pelo \textit{Kinect}.



A maior semelhança entre esse trabalho e os descritos anteriormente é que todos utilizam o \textit{Kinect} como um dispositivo substituto dos \textit{scanners} 3D tradicionais. Porém, ao contrário dos demais, este não utiliza as imagens RGB. Além disso, os trabalhos citados apenas executam um algoritmo de decisão sobre uma base de dados contendo imagens de mapas de profundidade gerados pelo  \textit{Kinect}, enquanto o objetivo desse trabalho é executar e exibir em  \textit{runtime} o resultado do algoritmo de detecção.
