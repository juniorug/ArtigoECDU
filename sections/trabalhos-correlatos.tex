\section{Trabalhos Relacionados}\label{sec:trabalhos-relacionados}

A pesquisa sobre o monitoramento de vídeo dinâmico tornou-se mais madura nos últimos anos. Além disso, alguns estudos de posicionamento combinaram Identificação por radiofrequência (\textit{Radio-Frequency IDentification - RFID}) e monitoramento de vídeo. Germa, T. et al. \cite{germa2010vision} propôs\textcolor{red}{et al. => plural / propuseram?} um sistema para identificar as pessoas que utilizam RFID e monitoramento de vídeo. Eles validaram a viabilidade da combinação de imagens com técnicas de RFID, e projetaram um sistema de rastreamento e identificação de pessoas. No entanto, este sistema exige um leitor de RFID e robô móvel funcionalmente equipado com um complicado sistema de  antenas multidireccionais, que tem um elevado custo.

Mandeljc et al. \cite{mandeljc2012tracking} propôs\textcolor{red}{idem} um sistema usando Banda \textcolor{red}{ultralarga} (\textit{Ultra wideband - UWB}) e vídeo para o posicionamento de pessoas em ambientes \textit{indoor}. Embora este sistema tenha um bom efeito de posicionamento, equipamentos UWB e múltiplas câmeras são obrigados a ser instalado no local, tornando o custo de construção do sistema é muito alto.

Wang et al. \cite{wang2011flexible,wang2011rfid} criaram um sistema de monitoramento em tempo real e posicionamento de pessoas em ambientes \textit{indoor} combinando imagem com  tecnologias RFID. No entanto, encontra-se experimental e continuam a verificar sobreposição de pessoas e problemas de blindagem, bem como os resultados de posicionamento são susceptíveis de serem influenciadas por altura e tipo do corpo, assim, o efeito implementação requer melhorias.

A utilização do Kinect resolve os problemas anteriormente referidos. Com o Kinect pode-se extrair com precisão as pessoas e a identificação não terá falhas resultantes de diferentes alturas, tipos de corpo e cores de pele. Seu o chip de processamento de imagem incorporado pode aumentar significativamente a eficiência do processamento de imagem, o que a torna muito aplicável à implementação de um sistema de posicionamento de pessoas \textit{indoor}.

Alguns estudos utilizam o Kinect para projetar sistemas de posicionamento de pessoas. Schindhelm \cite{schindhelm2012evaluating} \textcolor{red}{provou o mostrou?}provou que o Kinect foi muito aplicável no posicionamento \textit{indoor} de edifícios públicos. Nakano et al. \cite{nakano2012kinect} também propôs usar o Kinect para o posicionamento de pessoas \textit{indoor}. No entanto, os sistemas acima referidos não têm a função de identificação de pessoas. Este trabalho combina a capacidade de definir o posicionamento preciso de uma pessoa do Kinect com a função de identificação de pessoas do RFID para completar um sistema de posicionamento de interiores para o posicionamento preciso e identificação de pessoas.

\textcolor{red}{- carece de mais trabalhos relacionados para seu trabalho final - mais tarde é importante, uma tabela comparativa.}