\section{Referencial Teórico}\label{sec:referencial-teorico}

Nesta Seção, os principais conceitos relacionados a este trabalho são apresentados, fornecendo subsídios para o desenvolvimento do projeto de detecção de pessoas em ambiente fechado, usando mapas de profundidade. A subseção \ref{sec:deteccao-rastreamento} apresenta conceitos relacionados à detecção e rastreamento de pessoas. A subseção \ref{sec:tec-rastreamento} discute o tópico tecnologias de rastreamento. Por fim, na subseção \ref{sec:determ-posic} são abordadas as técnicas de determinação de posição comumente utilizadas. \textcolor{red}{REVER esse paragrafo, atualizando-o ao final da edição dessa seção. EX: tecnologias de rastreamento devem ser exluidas.}


\subsection{Biometria}\label{sec:biometria}
O termo biometria deriva do grego bios (vida) + metron (medida) e, na autenticação, refere-se à utilização de características próprias de um indivíduo para proceder à sua autenticação e/ou identificação \cite{magalhaes2003biometria}. Na biometria utiliza-se caracteristicas fisicas/fisiológicas ou comportamentais para a identificação de pessoas, sendo baseada em algo que a pessoa e e não em algo que ela possui ou sabe. A Figura~\ref{fig:biometria} ilustra exemplos de características biométricas \cite{cardia2015avaliaccao}.

\begin{figure*}[ht]
\centering
    \includegraphics[resolution=300,width=0.7\textwidth,natwidth=610,natheight=642]{images/biometria.png}
    \caption{Exemplos de características biométricas: impressão digital, orelha, termograma facial, termograma da mão, padrões de veias da mão, íris, retina, face e assinatura.}
    \label{fig:biometria}
\end{figure*}

Características fisicas/fisiológicas são baseadas na anatomia ou no funcionamento do organismo de uma pessoa viva, como impressão digitalou termograma facial. Já as comportamentais são baseadas na forma particular que um sujeito executa uma ação, como o a dinâmica de digitação ou a assinatura captada de modo digital. Qualquer característica pode ser utilizada desde que atenda algumas restrições, tais como \cite{maltoni2009handbook}:



\subsection{Detecção, identificação e rastreamento}\label{sec:deteccao-rastreamento}

As aplicações de localização interior têm experimentado esforços adicionais nos últimos anos com o aparecimento da computação ubíqua. Em vários cenários, objetos e mercadorias devem estar localizados ou monitorados, por exemplo, em um ambiente industrial ou médico. Adicionalmente, a localização de pessoas permite a criação de uma série de aplicativos e serviços. Clientes e funcionários podem ser observados, invasores detectados, idosos assistidos, e pacientes acompanhados. Em locais públicos, como estações de metrô, as preocupações de segurança podem ser abordadas por um sistema de orientação de emergência \cite{linde2006aspects}. 

Segundo Jaeseok Yun e Sang-Shin Lee \cite{yun2014human}, um sistema de controle de movimentos deve detectar robustamente: 
\begin{itemize}
  \item A identidade do objeto em movimento;
  \item Em qual o local está o objeto; 
  \item Em que sentido o objeto está se movimentando;
  \item O quão rápido esse objeto se move. 
 \end{itemize}

Uma das questões-chave da emergente computação móvel e robótica é a obtenção do conhecimento da posição de pessoas e objetos em um ambiente interno \cite{linde2006aspects}. A fim de construir um ambiente inteligente, onde os sistemas possam entender as atividades nas quais o usuário está envolvido e suas imediações, para então adaptar os seus serviços e recursos para o contexto do usuário, é necessário desenvolver um sistema de detecção, identificação e rastreamento de movimento robusto usando vários sensores \cite{yun2014human}. Esses três conceitos serão discutidos a seguir.


\subsubsection{Detecção de pessoas}\label{sec:deteccao-movimento}
 
 \textcolor{red}{essa subsecao está pobre}
 
\subsubsection{Identificação de indivíduos}\label{sec:identificacao-pessoas}

Sistemas de sensores para identificação recolhem um conjunto de dados brutos do corpo humano, bem como extraem características distintas visando reconhecer o contexto principal: a identidade do objeto \cite{yun2014human}. Para esse fim, inúmeros sistemas têm sido estudados usando vários sensores, incluindo câmeras, sensores de movimento, blocos de pressão, radares, sensores de campo elétrico, etc.
\textcolor{red}{essa subsecao está pobre}
 
\subsubsection{Rastreamento de pessoas}\label{sec:rastre-amb-fec}

 \textcolor{red}{essa subsecao está pobre}

\subsection{Tecnologias de detecção Visuais, não visuais e combinadas}\label{sec:tec-rastreamento}
No geral, sistemas de monitoramento podem ser não visuais, visuais (com ou sem a utilização de marcadores) ou uma combinação de ambos \cite{zhou2008human}.

\subsubsection{Tecnologias visuais}\label{sec:sens-genericos}
Muitos pesquisadores têm dedicado os seus esforços para a construção de sistemas de detecção de movimento robustos usando sensores baseados na visão usando câmeras. Os projetos de investigação com base em sensores baseados em visão consideram, principalmente, posição, velocidade, direção, forma e tamanho (ou seja, o número de \textit{pixels} em câmeras) como o contexto principal para identificar os usuários e compreender as suas atividades \cite{stauffer200l}.

Existem duas principais técnicas no acompanhamento visual do movimento humano: rastreamento baseado marcador e rastreamento livre de marcador \cite{YTao2010}.

Rastreamento visual com marcador base (\textit{Visual marker based tracking}) é uma técnica onde as câmeras são utilizadas para controlar os movimentos humanos. São adicionados identificadores sobre o corpo humano. Devido ao fato de o esqueleto humano ser uma estrutura altamente articulada, torções e rotações podem gerar movimento muito complexos. Como consequência, cada parte do corpo realiza uma trajetória de movimento imprevisível e complicada, o que pode levar a estimativa de movimento inconsistente e pouco confiável. Além disso, cenas desordenadas, ou variação de iluminação podem distrair a atenção visual a partir da posição real de um marcador. Como uma solução para esses problemas, o rastreamento visual com marcador base é preferível nestas circunstâncias \cite{zhang2002visual}.

Entretanto, essas tecnologias possuem limitações as quais podem apresentar falhas devido:

\begin{enumerate}
    \item A identificação dos pontos ósseos padrões pode não ser confiável; 
    \item O tecido macio que se sobrepõe pontos ósseos podem mover-se, dando origem a dados ruidosos; 
    \item O próprio marcador pode oscilar devido à sua própria inércia; 
    \item Marcadores podem mesmo vir à deriva completamente.
\end{enumerate}

Sistemas visuais de rastreamento livres de marcador (\textit{Marker-free visual based tracking systems}) somente exploram sensores óticos para medir o movimentos do corpo humano. Esta aplicação é motivada pelas falhas do uso de sistemas baseados em marcador visual, anteriormente apresentadas. 
 \textcolor{red}{essa subsubsubsubsubsecao está pobre}



\subsubsection{Tecnologias não visuais}\label{sec:tec-vis}
Sensores utilizados nestes sistemas interagem direta ou indiretamente com o corpo humano a fim de recolher informações relativas ao movimento. Estes sensores são comumente classificados como mecânicos, inerciais, envoltório acústico, rádio ou micro-ondas e com base magnética \cite{zhou2008human}. De um modo geral, cada tipo de sensor tem as suas próprias vantagens e limitações. Limitações de modalidade específica, medição específica, e circunstâncias específicas afetam, consequentemente, o uso de tipos particulares de sensores, em ambientes diferentes \cite{Welch:2002}.

\subsubsection{Tecnologias combinadas de rastreamento}\label{sec:rastre-robo}
Estes sistemas tiram proveito das vantagens das tecnologias visuais e não visuais. Essa estratégia de combinação ajuda a reduzir erros decorrentes da utilização de plataformas individuais. Por exemplo, os limites ou silhuetas de partes do corpo humano podem ser capturados em uma trajetória de movimento, se marcadores montados sobre essas partes não estão no "campo de visão" das câmeras. Essa estratégia exige calibração e computação intensivas \cite{zhou2008human}.

\subsection{Técnicas de determinação de posição}\label{sec:determ-posic}
Por natureza, o posicionamento é um problema interdisciplinar que traz consigo inúmeras questões em vários campos de pesquisa, como engenharia, ciência da computação e estatística. Como consequência, a concepção e implementação de um sistema de localização é uma tarefa bastante complexa que implica um entendimento nesses domínios \cite{linde2006aspects}. \textcolor{red}{por isso, vc deve delimitar bem seu trabalho, para nao correr o risco de prometer demais}

\subsubsection{Triangulação, proximidade e análise de cena}\label{sec:triang}

Triangulação requer a medição de ângulos entre as unidades de referência fixas e o alvo móvel. As unidades móveis calculam os ângulos em direção aos sinais emitidos por unidades fixas de referência. Obtém-se então a posição e orientação através dos dados recolhidos. As unidades de referência medem os ângulos na direção do sinal emitido pela unidade móvel. Apenas uma estimativa da localização, mas não a orientação do alvo móvel, pode ser obtida \cite{lutzke2013experimental}.

 \textcolor{red}{essa subsubsubsubsubsecao está pobre. DETALHAR triangulação, incluindo calculos.}


Outra abordagem de localização é a técnica de detecção de proximidade na qual a posição de um alvo é aproximada, selecionando a localização da unidade de referência mais próxima. Por conseguinte, não é necessário cálculo de localização. Alternativamente, se várias unidades de referência estão dentro do alcance, o baricentro entre estas unidades podem produzir uma melhor estimativa \cite{EHuber1996}.
 \textcolor{red}{essa subsubsubsubsubsecao está pobre. DETALHAR detecção de proximidade, incluindo calculos.}

Pontos de referência (\textit{Landmarks}) são elementos estáticos de um ambiente que pode ser reconhecido por uma unidade móvel. Na maioria das vezes, os pontos de referência são formas geométricas, como retângulos, linhas ou códigos de barras. Itens naturais, como portas podem também servir como pontos de referência. O termo  análise de cena é frequentemente utilizado neste contexto. Unidades móveis tentam localizar-se  em determinado ambiente usando a visão da câmera, bem como funcionalidades de extração para analisar o cenário atual. Uma vez que um marco foi reconhecido e identificado de forma confiável, a posição atual da unidade em relação ao marco correspondente pode ser calculada. Isto é conseguido através de triangulação, trilateração e proximidade \cite{linde2006aspects}.
 \textcolor{red}{ADICIONAR trilateração, incluindo calculos.}

\subsubsection{Posição física e localização simbólica}\label{sec:posic-fisica}

A localização simbólica, ou posicionamento relativo, descreve o procedimento de determinação da posição atual de um alvo móvel usando o curso e a velocidade da informação. Pode ser subdividida em duas abordagens: odometria e navegação inercial \cite{linde2006aspects}.

Odometria é uma abordagem avançada para estimar navegação. Começando a partir de uma posição conhecida, a presente localização de uma unidade pode então ser determinada por meio da reconstrução do caminho percorrido. A odometria é totalmente autossuficiente mas, por outro lado, está sujeita a erros. Esses erros podem ser sistemáticos, como por exemplo erros causados por diâmetros desiguais ou desalinhamento das rodas, ou não sistemáticos, como por exemplo movimento sobre solos não uniformes ou sobre obstáculos inesperados. 

Sistemas de navegação inercial (INS) usam giroscópios e acelerômetros para medir a velocidade de rotação e aceleração de uma unidade. A Informação sobre a posição é calculada mediante a integração dos dados medidos duas vezes. Assim como os sistemas de odometria, os sistemas de navegação inercial são autossuficientes, mas estão susceptíveis aos mesmos erros que podem ocorrer em sistemas de odometria. Um sistema inercial pode ajudar a compensar erros de odometria momentâneas \cite{linde2006aspects}.

A posição física, ou absoluta, de uma unidade móvel pode ser determinada com a ajuda de pontos de referência fixos localizados no ambiente. A posição destes pontos de referência é conhecida a priori e essas referências podem ser componentes ativos ou passivos. Esta técnica apresenta três abordagens: registro de balizas, ponto de referência e modelo de harmonização.
 \textcolor{red}{REQUER REFERÊNCIA!}

Balizas ativas são componentes estáticos localizados em posições fixas e conhecidas do ambiente. Existem dois tipos diferentes de balizas: balizas auto atuantes que emitem periodicamente certa assinatura (por exemplo, uma sequência de \textit{bits} única), e balizas sensíveis. Balizas sensíveis podem atuar como ouvintes ou ativamente refletir uma assinatura recebida emitida pela unidade móvel. Ponto de referência são características estáticas de um ambiente que podem ser reconhecidos por uma unidade móvel. Na maioria das vezes, esses marcos são formas geométricas, como retângulos, linhas ou códigos de barras. Além desses objetos artificiais, itens naturais, como portas também podem servir como pontos de referência. Além disso, a análise geral da cena é frequentemente utilizada neste contexto. No modelo de harmonização, uma unidade móvel deve ser capaz de construir um mapa ou modelo de um ambiente desconhecido e, ao mesmo tempo localizar-se no interior deste mapa. Enquanto se move e explora, um modelo de referência é criado. O posicionamento é realizado comparando este modelo de referência (possivelmente pré-armazenado) a um modelo local gerado a partir dos dados dos sensores a bordo \cite{linde2006aspects}.

\textcolor{red}{como que tudo isso que vc escreveu nessa secao se relacionada ao seu trabalho? o que vc usará? é sempre bom ir deixando claro isso.}


 Neste trabalho, será apresentada uma nova abordagem sensorial: associação de Técnicas Visuais e Não Visuais para a detecção \textcolor{red}{alinhe isso com o seu título. lá vc fala de identificacao e rastreamento, aqui vc diz deteccao. novamente misturando os conceitos. seja conciso nisso} de presença em um ambiente interno \cite{yun2014human}.