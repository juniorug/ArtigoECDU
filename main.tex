
%% bare_conf.tex
%% V1.3
%% 2007/01/11
%% by Michael Shell
%% See:
%% http://www.michaelshell.org/
%% for current contact information.
%%
%% This is a skeleton file demonstrating the use of IEEEtran.cls
%% (requires IEEEtran.cls version 1.7 or later) with an IEEE conference paper.
%%
%% Support sites:
%% http://www.michaelshell.org/tex/ieeetran/
%% http://www.ctan.org/tex-archive/macros/latex/contrib/IEEEtran/
%% and
%% http://www.ieee.org/


\documentclass[conference]{IEEEtran}
\usepackage{blindtext, graphicx}
\usepackage[justification=centering]{caption}
\usepackage[utf8]{inputenc}
\usepackage[T1]{fontenc}
\usepackage{amsmath}
\usepackage[english,portuguese]{babel}
\usepackage{hyphenat}
\hyphenation{mate-mática recu-perar}
\hyphenation{op-tical net-works semi-conduc-tor}
\usepackage{verbatim}
\usepackage{rotating}
\usepackage{latexsym}
\usepackage{amssymb}
\usepackage{graphicx,url}
\usepackage{color}


\begin{document}

%\title{Identificação e Rastreamento de Pessoas em Ambientes Internos Utilizando Humanos Como Sensores}
\title{Utilizando Mapas de Profundidade para Detecção de Pessoas em Ambientes Internos}

\author{
\IEEEauthorblockN{Edivaldo M. F. de Jesus Jr.}
\IEEEauthorblockA{Especialização em Computação Distribuída e Ubíqua\\
Grupo de Sistemas Distribuídos, Otimização, Redes e \\
Tempo-Real - GSORT\\
edivaldojunior@ifba.edu.br}

\and
\IEEEauthorblockN{Manoel C. M. Neto}
\IEEEauthorblockA{Especialização em Computação Distribuída e Ubíqua\\
Grupo de Sistemas Distribuídos, Otimização, Redes e \\
Tempo-Real - GSORT\\
manoelnetom@ifba.edu.br}
}

\maketitle

{\selectlanguage{english}
\begin{abstract}
The Advancement of location technologies has been providing grants for creating efficient and robust applications. Due to the fact that people spend most of their time in indoors environments, these indoor tracking services are in great demand from the public. Although various approaches have been proposed to deal with this problem, solutions still remain untackled due to various reasons (e.g. user acceptance). Based on this observation, this paper aims to provide a better understanding of the technologies currently used and stimulate a recent concept in sensory area: using depth maps generated by Microsoft Kinect for this purpose.
\end{abstract}}

\begin{abstract}
O avanço nas tecnologias de localização tem fornecido subsídios para criação de aplicações eficientes e robustas. Devido ao fato de que as pessoas passam a maior parte do seu tempo em ambientes fechados, serviços de detecção e rastreamento estão em grande demanda do público. Embora tenham sido propostas várias abordagens para lidar com este problema, as soluções ainda permanecem não resolvidas. Problemas como pose, iluminação e aceitação do usuário são alguns dos desafios complexos para o reconhecimento de faces 2D. Com base nessa observação, este trabalho tem como objetivo proporcionar uma melhor compreensão das tecnologias atualmente utilizadas e estimular um recente conceito na área sensorial: a utilização de mapas de profundidade. Menos intrusivo, se comparado à soluções com cameras RGB e custos muito abaixo dos scanners 3D tradicionais, o Microsoft Kinect é capaz de fornecer esse mapa de profundidade que é entao processado por um algoritmo e, em caso de detecção, os dados são exportados para que plataformas externas possam consumí-los. Dados obtidos foram comparados com outros mecanismos de detecção, e os resultados apresentados.  
\end{abstract}

{\selectlanguage{portuguese}
\begin{IEEEkeywords}
Detecção de pessoas, localização, ambientes internos.
\end{IEEEkeywords}}

\IEEEpeerreviewmaketitle


%Seções
\section{Introdução}\label{sec:introducao}

Detecção e rastreamento de pessoas são problemas desafiadores, especialmente em cenários reais e complexos. Esses cenários normalmente envolvem várias pessoas, obstruções e origens desordenadas ou ainda cenários com plano de fundo em movimento. Detectores de pessoas têm se mostrado capazes de localizar os pedestres mesmo em cenas complexas de rua. Entretanto, falsos positivos são frequentemente encontrados \cite{andriluka2008people}. A identificação de indivíduos particulares permanece igualmente desafiadora.
Métodos de rastreamento são capazes de encontrar um indivíduo em particular em sequências de imagens, mas são severamente desafiados por cenários do mundo real, tais como cenas de uma rua lotada \cite{andriluka2008people}.

Para ambientes abertos, sistemas de GPS (\textit{Global Positioning System}) desempenham um papel dominante no que se refere à localização \cite{fritsche2009}. No entanto, essa tecnologia não funciona bem em ambientes fechados. Essa ineficiência é consequência da fraqueza de sinais emitidos por GPS e a sua incapacidade de penetrar a maioria dos materiais de construção. Portanto, GPS não se encaixa bem em ambientes internos onde as pessoas passam a maior parte do seu tempo. Mesmo que os dispositivos GPS se tornem mais promissores e ponderáveis no futuro, sendo capazes de proporcionar uma precisão suficiente para uso ao ar livre, tecnologias mais eficazes são exigidas para o rastreamento de humanos e objetos em ambientes interiores \cite{zhang2010localization}.

Devido à complexidade desses ambientes, o desenvolvimento de uma técnica de localização interna é sempre acompanhado de um conjunto de desafios, como por exemplo: objetos fora do ângulo de visão (\textit{non line of sight} - NLOS); efeito de caminhos múltiplos; e interferência de ruído. Esses desafios resultam principalmente da influência de obstáculos (paredes, equipamentos, e/ou seres humanos) na propagação de ondas eletromagnéticas. A mobilidade das pessoas incorre mudanças nas condições físicas do ambiente, o que pode afetar de forma significativa o comportamento da propagação de sinal \textit{wireless}, por exemplo \cite{zhang2010localization}. 

Um sistema de localização interior, como definido por Dempsey \cite{dempsey2003indoor}, é um sistema que pode determinar a posição de algo ou alguém em um espaço físico, como em um hospital, um ginásio, uma escola, etc., de forma contínua e em tempo real. Porém, rastrear e acompanhar o movimento de pessoas em ambientes internos é útil para uma variedade de aplicações, incluindo cuidados a bebês e idosos, estudo do comportamento do cliente nos \textit{shopping centers}, segurança, etc. \cite{yiu2007tracking}.

Sensores de infravermelho passivos (PIR- \textit{Passive infrared}) são comumente usados em conjunto com uma variedade de outros sensores em diversas aplicações para a construção de ambientes inteligentes, como saúde, sistema de energia inteligente e segurança \cite{yun2014human}.

O conceito de Casa inteligente (\textit{Smart Home}) compreende vários serviços associados para cumprir objetivos diferentes. Sua finalidade é reforçar as atividades rotineiras dos habitantes, facilitando a vida independente de pessoas com deficiência, pacientes e idosos residentes \cite{al2014advanced}. Esses serviços podem ser divididos em seis categorias: conforto, gestão de energia, multimídia e entretenimento, saúde, segurança e proteção, e comunicações \cite{dewsbury2001process}.

Porém, criar uma abordagem de detecção e identificação humana é um problema desafiador, devido à complexidade das condições ambientais de espaços inteligentes e a variedade de serviços e demandas dos usuários \cite{al2014advanced}.

O processo de localização inteiro é dividido em duas fases: a medição de sinal e de cálculo de posição \cite{zhang2010localization}. 

Nos últimos anos vários métodos foram propostos para detecção humana \cite{dalal2005,dalal2006,ikemura2011,schwartz2009}. A maior parte da pesquisa é feita baseada em imagens extraídas de câmeras de luz visível (Câmeras RGB), que é uma maneira natural de fazê-la, exatamente como os olhos humanos funcionam. Alguns métodos envolvem o treino estatístico com base em características locais, tais como Histograma de Gradiente Orientado (\textit{Histograms of Oriented gradient} – HOG) \cite{dalal2005}., Histogramas de Orientação de Borda (\textit{Edge Orientation Histograms} – EOH) \cite{levi2004}, e alguns envolvem a extração de pontos de interesse na imagem, tais como transformada de característica invariante em escala (\textit{Scale-invariant Feature Transform} - SIFT) \cite{lowe1999}, etc.

Embora muitos relatórios tenham mostrado que esses métodos podem fornecer resultados de detecção altamente precisos, os métodos baseados em imagens RGB encontram dificuldades em perceber as formas dos sujeitos com poses articuladas ou quando o plano de fundo está incompreensível \cite{xia2011human}. Isso resultará na queda da precisão ou no aumento do custo computacional. As informações de profundidade são uma sugestão importante quando o ser humano reconhece objetos porque os objetos podem não ter cor e textura consistentes, mas devem ocupar uma região integrada no espaço \cite{xia2011human}. 

É notavelmente crescente nas últimas décadas pesquisas utilizando imagem de profundidade para o reconhecimento ou modelagem de objetos \cite{sabata1993, vemuri1986}. As imagens de profundidade têm várias vantagens sobre as imagens de intensidade 2D, dentre elas: as imagens de profundidade são resistentes à mudança de cor e de iluminação. Além disso, as imagens de alcance são representações simples de informações 3D. Contudo, os sensores de alcance de profundidade são caros e difíceis de utilizar em ambientes habitados devido à utilização de lasers. Uma alternativa é a utilização do Microsoft Kinect, o qual além de ser mais barato e fácil de usar, se comparado aos scanners 3D convencionais, não tem as desvantagens da utilização de laser\cite{xia2011human}.

Apesar do grande progresso nos últimos anos, há uma série de questões em aberto que precisam ser abordadas. Exemplos incluem rastreamento contínuo de pessoas que circulam entre ambientes internos e externos, a resolução de problemas de sincronização, reduzindo o impacto da interferência de ruído e melhorar a eficiência energética. Embora algumas tecnologias anteriores estejam preocupados com estas questões, elas podem sofrer de várias limitações, por exemplo, em aumentar o custo de todo o sistema, a deficiência de precisão, e sobrecarga computacional \cite{zhang2010localization}.

Este trabalho utiliza mapas de profundidade gerados pelo Kinect com o objetivo de detectar pessoas em um ambiente fechado.

Além desta Introdução, este trabalho está estruturado da seguinte forma:
A Seção \ref{sec:referencial-teorico} apresenta várias tecnologias envolvidas na preparação deste trabalho. A Seção \ref{sec:trabalhos-relacionados} apresenta os trabalhos correlatos. A Seção \ref{sec:solucao-desenvolvida} apresenta detalhes da implementação a ser desenvolvida. A Seção \ref{sec:conclusao} conclui este trabalho, destacando as suas limitações e trabalhos futuros.
\section{Referencial Teórico}\label{sec:referencial-teorico}

Nesta Seção, os principais conceitos relacionados a este trabalho são apresentados, fornecendo subsídios para o desenvolvimento do projeto de detecção de pessoas em ambiente fechado, usando mapas de profundidade. A subseção \ref{sec:deteccao-rastreamento} apresenta conceitos relacionados à detecção e rastreamento de pessoas. A subseção \ref{sec:tec-rastreamento} discute o tópico tecnologias de rastreamento. Por fim, na subseção \ref{sec:determ-posic} são abordadas as técnicas de determinação de posição comumente utilizadas. \textcolor{red}{REVER esse paragrafo, atualizando-o ao final da edição dessa seção. EX: tecnologias de rastreamento devem ser exluidas.}


\subsection{Biometria}\label{sec:biometria}
O termo biometria deriva do grego bios (vida) + metron (medida) e, na autenticação, refere-se à utilização de características próprias de um indivíduo para proceder à sua autenticação e/ou identificação \cite{magalhaes2003biometria}. Na biometria utiliza-se caracteristicas fisicas/fisiológicas ou comportamentais para a identificação de pessoas, sendo baseada em algo que a pessoa e e não em algo que ela possui ou sabe. A Figura~\ref{fig:biometria} ilustra exemplos de características biométricas \cite{cardia2015avaliaccao}.

\begin{figure*}[ht]
\centering
    \includegraphics[resolution=300,width=0.7\textwidth,natwidth=610,natheight=642]{images/biometria.png}
    \caption{Exemplos de características biométricas: impressão digital, orelha, termograma facial, termograma da mão, padrões de veias da mão, íris, retina, face e assinatura.}
    \label{fig:biometria}
\end{figure*}

Características fisicas/fisiológicas são baseadas na anatomia ou no funcionamento do organismo de uma pessoa viva, como impressão digitalou termograma facial. Já as comportamentais são baseadas na forma particular que um sujeito executa uma ação, como o a dinâmica de digitação ou a assinatura captada de modo digital. Qualquer característica pode ser utilizada desde que atenda algumas restrições, tais como \cite{maltoni2009handbook}:



\subsection{Detecção, identificação e rastreamento}\label{sec:deteccao-rastreamento}

As aplicações de localização interior têm experimentado esforços adicionais nos últimos anos com o aparecimento da computação ubíqua. Em vários cenários, objetos e mercadorias devem estar localizados ou monitorados, por exemplo, em um ambiente industrial ou médico. Adicionalmente, a localização de pessoas permite a criação de uma série de aplicativos e serviços. Clientes e funcionários podem ser observados, invasores detectados, idosos assistidos, e pacientes acompanhados. Em locais públicos, como estações de metrô, as preocupações de segurança podem ser abordadas por um sistema de orientação de emergência \cite{linde2006aspects}. 

Segundo Jaeseok Yun e Sang-Shin Lee \cite{yun2014human}, um sistema de controle de movimentos deve detectar robustamente: 
\begin{itemize}
  \item A identidade do objeto em movimento;
  \item Em qual o local está o objeto; 
  \item Em que sentido o objeto está se movimentando;
  \item O quão rápido esse objeto se move. 
 \end{itemize}

Uma das questões-chave da emergente computação móvel e robótica é a obtenção do conhecimento da posição de pessoas e objetos em um ambiente interno \cite{linde2006aspects}. A fim de construir um ambiente inteligente, onde os sistemas possam entender as atividades nas quais o usuário está envolvido e suas imediações, para então adaptar os seus serviços e recursos para o contexto do usuário, é necessário desenvolver um sistema de detecção, identificação e rastreamento de movimento robusto usando vários sensores \cite{yun2014human}. Esses três conceitos serão discutidos a seguir.


\subsubsection{Detecção de pessoas}\label{sec:deteccao-movimento}
 
 \textcolor{red}{essa subsecao está pobre}
 
\subsubsection{Identificação de indivíduos}\label{sec:identificacao-pessoas}

Sistemas de sensores para identificação recolhem um conjunto de dados brutos do corpo humano, bem como extraem características distintas visando reconhecer o contexto principal: a identidade do objeto \cite{yun2014human}. Para esse fim, inúmeros sistemas têm sido estudados usando vários sensores, incluindo câmeras, sensores de movimento, blocos de pressão, radares, sensores de campo elétrico, etc.
\textcolor{red}{essa subsecao está pobre}
 
\subsubsection{Rastreamento de pessoas}\label{sec:rastre-amb-fec}

 \textcolor{red}{essa subsecao está pobre}

\subsection{Tecnologias de detecção Visuais, não visuais e combinadas}\label{sec:tec-rastreamento}
No geral, sistemas de monitoramento podem ser não visuais, visuais (com ou sem a utilização de marcadores) ou uma combinação de ambos \cite{zhou2008human}.

\subsubsection{Tecnologias visuais}\label{sec:sens-genericos}
Muitos pesquisadores têm dedicado os seus esforços para a construção de sistemas de detecção de movimento robustos usando sensores baseados na visão usando câmeras. Os projetos de investigação com base em sensores baseados em visão consideram, principalmente, posição, velocidade, direção, forma e tamanho (ou seja, o número de \textit{pixels} em câmeras) como o contexto principal para identificar os usuários e compreender as suas atividades \cite{stauffer200l}.

Existem duas principais técnicas no acompanhamento visual do movimento humano: rastreamento baseado marcador e rastreamento livre de marcador \cite{YTao2010}.

Rastreamento visual com marcador base (\textit{Visual marker based tracking}) é uma técnica onde as câmeras são utilizadas para controlar os movimentos humanos. São adicionados identificadores sobre o corpo humano. Devido ao fato de o esqueleto humano ser uma estrutura altamente articulada, torções e rotações podem gerar movimento muito complexos. Como consequência, cada parte do corpo realiza uma trajetória de movimento imprevisível e complicada, o que pode levar a estimativa de movimento inconsistente e pouco confiável. Além disso, cenas desordenadas, ou variação de iluminação podem distrair a atenção visual a partir da posição real de um marcador. Como uma solução para esses problemas, o rastreamento visual com marcador base é preferível nestas circunstâncias \cite{zhang2002visual}.

Entretanto, essas tecnologias possuem limitações as quais podem apresentar falhas devido:

\begin{enumerate}
    \item A identificação dos pontos ósseos padrões pode não ser confiável; 
    \item O tecido macio que se sobrepõe pontos ósseos podem mover-se, dando origem a dados ruidosos; 
    \item O próprio marcador pode oscilar devido à sua própria inércia; 
    \item Marcadores podem mesmo vir à deriva completamente.
\end{enumerate}

Sistemas visuais de rastreamento livres de marcador (\textit{Marker-free visual based tracking systems}) somente exploram sensores óticos para medir o movimentos do corpo humano. Esta aplicação é motivada pelas falhas do uso de sistemas baseados em marcador visual, anteriormente apresentadas. 
 \textcolor{red}{essa subsubsubsubsubsecao está pobre}



\subsubsection{Tecnologias não visuais}\label{sec:tec-vis}
Sensores utilizados nestes sistemas interagem direta ou indiretamente com o corpo humano a fim de recolher informações relativas ao movimento. Estes sensores são comumente classificados como mecânicos, inerciais, envoltório acústico, rádio ou micro-ondas e com base magnética \cite{zhou2008human}. De um modo geral, cada tipo de sensor tem as suas próprias vantagens e limitações. Limitações de modalidade específica, medição específica, e circunstâncias específicas afetam, consequentemente, o uso de tipos particulares de sensores, em ambientes diferentes \cite{Welch:2002}.

\subsubsection{Tecnologias combinadas de rastreamento}\label{sec:rastre-robo}
Estes sistemas tiram proveito das vantagens das tecnologias visuais e não visuais. Essa estratégia de combinação ajuda a reduzir erros decorrentes da utilização de plataformas individuais. Por exemplo, os limites ou silhuetas de partes do corpo humano podem ser capturados em uma trajetória de movimento, se marcadores montados sobre essas partes não estão no "campo de visão" das câmeras. Essa estratégia exige calibração e computação intensivas \cite{zhou2008human}.

\subsection{Técnicas de determinação de posição}\label{sec:determ-posic}
Por natureza, o posicionamento é um problema interdisciplinar que traz consigo inúmeras questões em vários campos de pesquisa, como engenharia, ciência da computação e estatística. Como consequência, a concepção e implementação de um sistema de localização é uma tarefa bastante complexa que implica um entendimento nesses domínios \cite{linde2006aspects}. \textcolor{red}{por isso, vc deve delimitar bem seu trabalho, para nao correr o risco de prometer demais}

\subsubsection{Triangulação, proximidade e análise de cena}\label{sec:triang}

Triangulação requer a medição de ângulos entre as unidades de referência fixas e o alvo móvel. As unidades móveis calculam os ângulos em direção aos sinais emitidos por unidades fixas de referência. Obtém-se então a posição e orientação através dos dados recolhidos. As unidades de referência medem os ângulos na direção do sinal emitido pela unidade móvel. Apenas uma estimativa da localização, mas não a orientação do alvo móvel, pode ser obtida \cite{lutzke2013experimental}.

 \textcolor{red}{essa subsubsubsubsubsecao está pobre. DETALHAR triangulação, incluindo calculos.}


Outra abordagem de localização é a técnica de detecção de proximidade na qual a posição de um alvo é aproximada, selecionando a localização da unidade de referência mais próxima. Por conseguinte, não é necessário cálculo de localização. Alternativamente, se várias unidades de referência estão dentro do alcance, o baricentro entre estas unidades podem produzir uma melhor estimativa \cite{EHuber1996}.
 \textcolor{red}{essa subsubsubsubsubsecao está pobre. DETALHAR detecção de proximidade, incluindo calculos.}

Pontos de referência (\textit{Landmarks}) são elementos estáticos de um ambiente que pode ser reconhecido por uma unidade móvel. Na maioria das vezes, os pontos de referência são formas geométricas, como retângulos, linhas ou códigos de barras. Itens naturais, como portas podem também servir como pontos de referência. O termo  análise de cena é frequentemente utilizado neste contexto. Unidades móveis tentam localizar-se  em determinado ambiente usando a visão da câmera, bem como funcionalidades de extração para analisar o cenário atual. Uma vez que um marco foi reconhecido e identificado de forma confiável, a posição atual da unidade em relação ao marco correspondente pode ser calculada. Isto é conseguido através de triangulação, trilateração e proximidade \cite{linde2006aspects}.
 \textcolor{red}{ADICIONAR trilateração, incluindo calculos.}

\subsubsection{Posição física e localização simbólica}\label{sec:posic-fisica}

A localização simbólica, ou posicionamento relativo, descreve o procedimento de determinação da posição atual de um alvo móvel usando o curso e a velocidade da informação. Pode ser subdividida em duas abordagens: odometria e navegação inercial \cite{linde2006aspects}.

Odometria é uma abordagem avançada para estimar navegação. Começando a partir de uma posição conhecida, a presente localização de uma unidade pode então ser determinada por meio da reconstrução do caminho percorrido. A odometria é totalmente autossuficiente mas, por outro lado, está sujeita a erros. Esses erros podem ser sistemáticos, como por exemplo erros causados por diâmetros desiguais ou desalinhamento das rodas, ou não sistemáticos, como por exemplo movimento sobre solos não uniformes ou sobre obstáculos inesperados. 

Sistemas de navegação inercial (INS) usam giroscópios e acelerômetros para medir a velocidade de rotação e aceleração de uma unidade. A Informação sobre a posição é calculada mediante a integração dos dados medidos duas vezes. Assim como os sistemas de odometria, os sistemas de navegação inercial são autossuficientes, mas estão susceptíveis aos mesmos erros que podem ocorrer em sistemas de odometria. Um sistema inercial pode ajudar a compensar erros de odometria momentâneas \cite{linde2006aspects}.

A posição física, ou absoluta, de uma unidade móvel pode ser determinada com a ajuda de pontos de referência fixos localizados no ambiente. A posição destes pontos de referência é conhecida a priori e essas referências podem ser componentes ativos ou passivos. Esta técnica apresenta três abordagens: registro de balizas, ponto de referência e modelo de harmonização.
 \textcolor{red}{REQUER REFERÊNCIA!}

Balizas ativas são componentes estáticos localizados em posições fixas e conhecidas do ambiente. Existem dois tipos diferentes de balizas: balizas auto atuantes que emitem periodicamente certa assinatura (por exemplo, uma sequência de \textit{bits} única), e balizas sensíveis. Balizas sensíveis podem atuar como ouvintes ou ativamente refletir uma assinatura recebida emitida pela unidade móvel. Ponto de referência são características estáticas de um ambiente que podem ser reconhecidos por uma unidade móvel. Na maioria das vezes, esses marcos são formas geométricas, como retângulos, linhas ou códigos de barras. Além desses objetos artificiais, itens naturais, como portas também podem servir como pontos de referência. Além disso, a análise geral da cena é frequentemente utilizada neste contexto. No modelo de harmonização, uma unidade móvel deve ser capaz de construir um mapa ou modelo de um ambiente desconhecido e, ao mesmo tempo localizar-se no interior deste mapa. Enquanto se move e explora, um modelo de referência é criado. O posicionamento é realizado comparando este modelo de referência (possivelmente pré-armazenado) a um modelo local gerado a partir dos dados dos sensores a bordo \cite{linde2006aspects}.

\textcolor{red}{como que tudo isso que vc escreveu nessa secao se relacionada ao seu trabalho? o que vc usará? é sempre bom ir deixando claro isso.}


 Neste trabalho, será apresentada uma nova abordagem sensorial: associação de Técnicas Visuais e Não Visuais para a detecção \textcolor{red}{alinhe isso com o seu título. lá vc fala de identificacao e rastreamento, aqui vc diz deteccao. novamente misturando os conceitos. seja conciso nisso} de presença em um ambiente interno \cite{yun2014human}.
\section{Trabalhos Relacionados}\label{sec:trabalhos-relacionados}

A pesquisa sobre o monitoramento de vídeo dinâmico tornou-se mais madura nos últimos anos. Além disso, alguns estudos de posicionamento combinaram Identificação por radiofrequência (\textit{Radio-Frequency IDentification - RFID}) e monitoramento de vídeo. Germa, T. et al. \cite{germa2010vision} propôs\textcolor{red}{et al. => plural / propuseram?} um sistema para identificar as pessoas que utilizam RFID e monitoramento de vídeo. Eles validaram a viabilidade da combinação de imagens com técnicas de RFID, e projetaram um sistema de rastreamento e identificação de pessoas. No entanto, este sistema exige um leitor de RFID e robô móvel funcionalmente equipado com um complicado sistema de  antenas multidireccionais, que tem um elevado custo.

Mandeljc et al. \cite{mandeljc2012tracking} propôs\textcolor{red}{idem} um sistema usando Banda \textcolor{red}{ultralarga} (\textit{Ultra wideband - UWB}) e vídeo para o posicionamento de pessoas em ambientes \textit{indoor}. Embora este sistema tenha um bom efeito de posicionamento, equipamentos UWB e múltiplas câmeras são obrigados a ser instalado no local, tornando o custo de construção do sistema é muito alto.

Wang et al. \cite{wang2011flexible,wang2011rfid} criaram um sistema de monitoramento em tempo real e posicionamento de pessoas em ambientes \textit{indoor} combinando imagem com  tecnologias RFID. No entanto, encontra-se experimental e continuam a verificar sobreposição de pessoas e problemas de blindagem, bem como os resultados de posicionamento são susceptíveis de serem influenciadas por altura e tipo do corpo, assim, o efeito implementação requer melhorias.

A utilização do Kinect resolve os problemas anteriormente referidos. Com o Kinect pode-se extrair com precisão as pessoas e a identificação não terá falhas resultantes de diferentes alturas, tipos de corpo e cores de pele. Seu o chip de processamento de imagem incorporado pode aumentar significativamente a eficiência do processamento de imagem, o que a torna muito aplicável à implementação de um sistema de posicionamento de pessoas \textit{indoor}.

Alguns estudos utilizam o Kinect para projetar sistemas de posicionamento de pessoas. Schindhelm \cite{schindhelm2012evaluating} \textcolor{red}{provou o mostrou?}provou que o Kinect foi muito aplicável no posicionamento \textit{indoor} de edifícios públicos. Nakano et al. \cite{nakano2012kinect} também propôs usar o Kinect para o posicionamento de pessoas \textit{indoor}. No entanto, os sistemas acima referidos não têm a função de identificação de pessoas. Este trabalho combina a capacidade de definir o posicionamento preciso de uma pessoa do Kinect com a função de identificação de pessoas do RFID para completar um sistema de posicionamento de interiores para o posicionamento preciso e identificação de pessoas.

\textcolor{red}{- carece de mais trabalhos relacionados para seu trabalho final - mais tarde é importante, uma tabela comparativa.}
\section{Solução desenvolvida}\label{sec:solucao-desenvolvida}
Motivado pela demanda de serviços de identificação de pessoas e pelo fato de que as soluções para tal problema ainda permanecem em aberto, este trabalho propõe a utilização de um método ainda recente na área sensorial: a utilização de mapas de profundidade. O mecanismo de identificação consiste em extrair esse mapa através do projetor infravermelho e do sensor monocromático, e analisar os dados obtidos, utilizando de filtros e algoritmos de segmentação. Por fim, é exibida graficamente uma visualização do mapa de profundidade gerado e, em caso de positivo, o(s) indivíduo(s) encontrado(s) na cena. É exibido também os dados que podem ser exportados para plataformas externas: a localização aproximada dos sujeitos no plano 3D (posições $X$, $Y$ e $Z$) e a distância em relação ao \textit{Kinect}. Nesta seção é discutida a solução desenvolvida.



\subsection{Arquitetura da solução}\label{sec:arqSol}
O \textit{Kinect} será responsável por obter a visualização da cena de um ambiente interno. Como discutido na seção \ref{sec:trabalhos-relacionados}, o \textit{Kinect} é um sensor de movimento muito eficiente, extraindo com precisão objetos, pessoas e detalhes de uma cena. A \textit{Microsoft} provê para o \textit{Kinect} um Kit de desenvolvimento de software (\textit{Software Development Kit}, SDK).O sistema é composto por módulos adicionados à camada de aplicação, Sobre a camada do SDK. A Figura \ref{fig:visao-geral} apresenta uma visão geral da solução.
 
\begin{figure*}[ht]
\centering
\includegraphics[width=1.0\textwidth]{images/Arquitetura_da_solucao.png}
\caption{Arquitetura da solução proposta}
\label{fig:visao-geral}
\end{figure*}


\subsubsection{Estilos arquiteturais}\label{sec:estilosArq}
A solução apresenta um estilo arquitetural híbrido, com características de três estilos arquiteturais básicos:

\begin{itemize}
\item Baseado em invocação implícita: \textit{Event Based}; 
\item Baseado em fluxo de dados: \textit{Pipe and Filter}; 
\item Em camadas: \textit{Virtual Machine}.
\end{itemize}

O estilo baseado em eventos (\textit{Event Based}) é um estilo arquitetônico baseado na invocação implícita que fornece uma interação indireta entre componentes acoplados de forma livre, facilitando a adaptação e melhorando a escalabilidade do sistema. Os componentes do tipo de evento (\textit{Depth}, \textit{color} e \textit{Infrared}.) comunicam-se apenas através de eventos transmitidos por um conector de evento.
Este conector então retransmite os eventos para todos os componentes do tipo \textit{Observer} mostrando interesse no evento em questão (Nesse caso, o \textit{Recognition Stream}), melhorando assim a eficiência da distribuição de eventos.


\textit{Pipes} e Filtros (\textit{Pipe and Filter}) é um estilo arquitetural composto por uma cadeia de elementos de processamento, dispostos de forma tal que a saída de cada elemento é a entrada do próximo. O fluxo de dados se dá através de \textit{pipes} (canos) e os dados sofrem transformações quando processados nos filtros. Em outras palavras, os \textit{pipes} é que possibilitam o fluxo dos dados, e os filtros fazem o processamento dos mesmos, colocando-os nos \textit{pipes} antes que todos os dados de entrada sejam consumidos. Portanto, a nível de arquitetura, o processamento é mapeado em filtros e os \textit{pipes} agem como condutores de dados. O componente \textit{Recognition Stream} é composto por subcomponentes do tipo \textit{Filter} e a comunicação entre os mesmos se dá através dos conectores do tipo \textit{Pipe}. 

As arquiteturas de máquinas virtuais (\textit{Virtual Machines}) têm como objetivo alcançar a qualidade da portabilidade. Este estilo de arquitetural simula algumas funcionalidades que não são originais para o \textit{hardware} e/ou \textit{software} em que é implementado. O estilo é aplicado entre o \textit{Kinect (hardware)} e a aplicação desenvolvida, usando conectores do tipo \textit{Event} e \textit{Pipe} entre as camadas. Esse estilo arquitetônico reduz a complexidade, melhora a modularidade, reutilização e manutenção.
Conforme visto na figura \ref{fig:visao-geral}, o sistema possui as camadas:

\begin{itemize}
\item \textit{Kinect (hardware)};
\item \textit{Kinect} SDK; 
\item Aplicação; 
\end{itemize}


\subsubsection{Arquitetura do \textit{Kinect for Windows} SDK}\label{sec:kinectSDK}
O SDK fornece uma biblioteca e ferramentas de software sofisticadas para ajudar os desenvolvedores a usar a forma rica de entrada natural baseada no uso do \textit{Kinect}, que detecta e reage a eventos do mundo real. O \textit{Kinect} e a biblioteca de software interagem com aplicações, como mostrado na Figura \ref{fig:sdk_interact}. Os componentes do SDK são exibidos na figura \ref{fig:sdk_architecture_color} e incluem:

\begin{figure}[h]
\centering
\includegraphics[width=0.5\textwidth]{images/sdk_interaction.png}
\caption{Interação de Hardware e Software com a Aplicação}
\label{fig:sdk_interact}
\end{figure}


\begin{figure*}[ht]
\centering
\includegraphics[width=0.8\textwidth]{images/sdk_architecture_color.png}
\caption{Arquitetura do SDK}
\label{fig:sdk_architecture_color}
\end{figure*}

\begin{enumerate}
    \item \textit{Hardware Kinect} - Os componentes de \textit{hardware}, incluindo o sensor \textit{Kinect} e o \textit{hub} USB através dos quais o sensor \textit{Kinect} está conectado ao computador;
    \item \textit{Drivers Kinect} - Os \textit{drivers} do Windows para o \textit{Kinect}, que são instalados como parte do processo de configuração do SDK. Os \textit{drivers} do \textit{Kinect} suportam:
        \begin{itemize}
            \item A matriz de microfones como um dispositivo de áudio em modo \textit{kernel} que pode ser acessado  através das API de áudio padrão no Windows; 
            \item Controles de transmissão de áudio e vídeo para transmissão de áudio e vídeo (cor, profundidade e esqueleto);
            \item Funções de enumeração de dispositivos que permitem que um aplicativo use mais de um \textit{Kinect}.	
        \end{itemize} 
    \item Componentes de Áudio e Vídeo:
        \begin{itemize}
            \item Interface de usuário natural do Kinect para rastreamento de esqueleto, áudio e imagem em cores e profundidade. 
        \end{itemize}
    \item Objeto de mídia DirectX (DMO) para formatação de feixes de microfone e localização de fonte de áudio;
    \item API padrão do Windows - APIs de áudio, voz e mídia no Windows, conforme descrito no SDK do Windows e no Microsoft \textit{Speech} SDK.
\end{enumerate}


\subsection{Componentes principais}\label{sec:componentes}

\subsubsection{Color Data Stream}\label{sec:colorDataStream}

\subsubsection{Depth Data Stream}\label{sec:depthDataStream}
capturing data stream

\subsubsection{Skeleton recognizer }\label{sec:skeleton}
processing joints

\subsubsection{Depth Data Processing (Recognition Stream - Filter)}\label{sec:depthDataProcessing}
processing data stream



\subsubsection{Face Recognition (Recognition Stream )}\label{sec:depthDataRecognition}
processing data stream
%\section{Avaliação Experimental}\label{sec:avaliacao-experimental}

Esta seção existe apenas no TCC. Ela não deve existir no pré-projeto.

Aqui será apresentado a estudo de avaliação conduzido para valiar a solução desenvolvida.
\section{Métodos e Materiais}\label{sec:metodos-e-materiais}


Será realizado um estudo de caso, baseado em pesquisa experimental pois há um alto nível de controle da situação e podem-se isolar todas as estruturas de qualquer interferência do meio exterior, gerando maior confiabilidade nos resultados.

Será realizada uma análise quantitativa da quantidade de coincidências entre trilhas e detecções de forma a avaliar e validar a eficiência em relação à detecção de falsos positivos indicados pela solução. Esses dados serão obtidos fazendo um comparativo entre o cenário real e os dados indicados pela solução. Será utilizada para isso a técnica de estudo experimental \textit{In Virtuo}

A avaliação se dará no que segue:

\begin{enumerate}
\item Adicionar o Kinect e os sensores em pontos estratégicos da sala;
\item Iniciar o sistema produto da solução;
\item Avaliar entrada e saída de um grupo distinto de pessoas em momentos distintos e ao mesmo tempo;
\item Comparar os dados obtidos pela solução com o ocorrido nas cenas durante a avaliação
\item Calcular a eficiência de acordo com os dados obtidos dessa comparação.
\end{enumerate}


%\subsection{Passo a Passo Latex}\label{sec:passo-a-passo}
Esta subseção apresenta o passo a passo que deve ser seguido para utilizar o sharelatex com o template da ECDU.

\begin{enumerate}
\item Criar projeto no ShareLatex com template IEEE Conference
\item Coloque as suas bibliografias em um arquivo separado. Para isso crie o arquivo .bib. Veja exemplo "\\bibliography{references.bib}
\\bibliographystyle{IEEEtran}" no texto. Remova a parte de referencias que já vem no template
\item No lugar dos dois itens anteriores, você pode fazer o download do projeto ECDU-TCC-Modelo e depois fazer o upload para sua conta, ou fazer a cópia direta (Opção Copy Project no menu esquerdo superior).
\item Mudar para português. Veja o código necessário aqui https://pt.sharelatex.com/learn/Portuguese. Esse código já está disponível no arquivo main.tex. Ele está entre os comentários "Início do código para entender português" e "Fim do código para entender português"
\item Corretor Ortográfico para português. Mude as configurações do seu projeto para Português do Brasil: Menu -> Spell Check -> Portuguese (Brazilian)
\end{enumerate}


\section{Conclusão}\label{sec:conclusao}

Sistemas e aplicações de localização interior tem experimentado nos últimos anos esforços adicionais com o aparecimento da computação ubíqua. Em vários cenários, objetos e pessoas precisam ser localizados ou monitorados. Por exemplo, em uma indústria ou ambiente médico. Além disso, a localização das pessoas viabiliza a criação de um grande número de aplicações e serviços. Clientes e funcionários podem ser observados, intrusos detectados, idosos suportados e pacientes monitorados. Em locais públicos, como estações de metrô, a segurança pode ser abordada por um sistema de orientação de emergência. Sistemas que utilizam as informações de orientação alimentada com o posicionamento dos seres humanos podem trabalhar de forma mais eficiente do que um sistema estático. 

O presente trabalho implementa um sistema que utiliza técnicas visuais com o objetivo de detectar pessoas em um ambiente fechado. Para tal, foi utilizado o \textit{Kinect} como sensor visual. A ferramenta desenvolvida utiliza o mapa de profundidade gerado pelo sistema de câmeras do \textit{Kinect} aplicando sobre essa fonte de dados algoritmos de filtro e segmentação visando a correta identificação de indivíduos em determinado âmbito.
Com essas características, as principais contribuições são:

\begin{enumerate}
    \item Baixo custo de \textit{hardware}, se comparado o \textit{Kinect} com os principais \textit{scanners} 3D do mercado;  
    \item Exportação de dados de localização dos sujeitos dentro do ambiente monitorado para que plataformas externas possam consumí-los;
    \item Eficiência, se comparado com projetos de detecção que utilizam técnicas visuais 2D e abordagens híbridas;
    \item \textit{less intrusiveness}, pois apenas o mapa de profundidade é exibido, se compararmos com sistemas que utilizam câmeras RGB. 
\end{enumerate}

Sua estrutura arquitetural dividida em camadas e módulos, seguindo o estilo \textit{Pipe and Filter} permite a fácil alteração dos componentes, sem causar impacto nos outros  elementos da solução. 


\textcolor{red}{ADICIONAR AQUI OS RESULTADOS OBTIDOS}



\subsection{Limitações deste Trabalho}\label{sec:limitacoes}
Este trabalho não contempla a avaliação da velocidade nem direção/sentido ao qual pessoas se movimentam, limitando-se a identificar a posição do objeto de estudo num determinado ambiente.
Outra limitação refere-se ao limite de distância ao qual o \textit{Kinect} é capaz de identificar objetos e pessoas, pois a versão um do mesmo possui \textit{range} entre 0,8 e 4 metros. Ainda, uma outra limitação do hardware em uso é a quantidade máxima de seis pessoas a serem identificadas, por ambiente.

\subsection{Trabalhos Futuros}
Como trabalhos futuros, podem-se aplicar outras técnicas de processamento e detecção de pessoas, estimando e comparando o nível de eficácia da solução. Outra abordagem seria a adição de sensores não visuais, como por exemplo pisos táteis ou sensores ultrassônicos.

Para as restrições do \textit{hardware} citados na subseção \ref{sec:limitacoes} pode-se tanto adicionar outros \textit{Kinects} quanto substituir o \textit{Kinect} V1 pela segunda versão do mesmo, melhorando assim a Acurácia, precisão e \textit{range} da solução.



%\appendices
%\onecolumn
\section{DrawFaceModel}\label{sec:apendiceA}
\begin{minted}{csharp}
public void DrawFaceModel(DrawingContext drawingContext, int personIndex)
{

    if (!this.lastFaceTrackSucceeded || this.skeletonTrackingState != 
        SkeletonTrackingState.Tracked)
    {
        return;
    }

    var faceModelPts = new List<Point>();
    var faceModel = new List<FaceModelTriangle>();

    for (int i = 0; i < this.facePoints.Count; i++)
    {
        faceModelPts.Add(new Point(this.facePoints[i].X + 0.5f, 
          this.facePoints[i].Y + 0.5f));
    }

    foreach (var t in faceTriangles)
    {
        var triangle = new FaceModelTriangle();
        triangle.P1 = faceModelPts[t.First];
        triangle.P2 = faceModelPts[t.Second];
        triangle.P3 = faceModelPts[t.Third];
        faceModel.Add(triangle);

    }

    var faceModelGroup = new GeometryGroup();
    for (int i = 0; i < faceModel.Count; i++)
    {
        var faceTriangle = new GeometryGroup();
        faceTriangle.Children.Add(new LineGeometry(faceModel[i].P1, faceModel[i].P2));
        faceTriangle.Children.Add(new LineGeometry(faceModel[i].P2, faceModel[i].P3));
        faceTriangle.Children.Add(new LineGeometry(faceModel[i].P3, faceModel[i].P1));
        faceModelGroup.Children.Add(faceTriangle);
    }

    SolidColorBrush[] colors = new SolidColorBrush[6] { Brushes.Blue, Brushes.Yellow, 
        Brushes.Pink, Brushes.Green, Brushes.Pink, Brushes.Salmon };
    drawingContext.DrawGeometry(colors[(personIndex % 6)], 
    new Pen(colors[(personIndex % 6)], 1.0), faceModelGroup);
}
\end{minted}



% use section* for acknowledgement
%\section*{Acknowledgment}

%Os autores deste trabalho gostariam de agradecer ao ...


% Can use something like this to put references on a page
% by themselves when using endfloat and the captionsoff option.
\ifCLASSOPTIONcaptionsoff
  \newpage
\fi


% biography section

\bibliography{references.bib}
\bibliographystyle{IEEEtran}


% that's all folks
\end{document}


